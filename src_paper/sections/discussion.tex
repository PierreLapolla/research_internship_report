The results of our adapted model are promising but also reveal areas that require further research.
\begin{itemize}
    \item \textbf{Car Traffic Prediction}:
    Our model achieve a similar performance than the DCRNN baseline in terms of MAE and RMSE\@.
    This suggests that our modifications are not penalizing the model's ability to handle car traffic data.
    \item \textbf{Wheelchair Traffic Prediction}:
    There is no baseline for wheelchair traffic prediction, as this is a novel contribution of our study.
    The results are promising and suggest that the model can effectively handle wheelchair data.
    The losses are very low compared to the car traffic prediction, this indicates better accuracy but needs to be
    contextualized.
    As our speed feature is not normalized and wheelchair speeds are significantly lower than car speeds, losses are
    expected to be lower.
    \item \textbf{Multi-Modal Traffic Prediction}:
    The model's performance on multi-modal traffic prediction, combining cars and wheelchair data, is not as good as
    expected.
    This confirms the complexity of handling multiple locomotion modes with a single model.
    The increase of the MAE and RMSE could come from the data and/or the model architecture.
\end{itemize}
\vspace{1em}

Our study introduces some implications and limitations, a first implication is data derivation.
Although it offers a new methodology, the process of obtaining wheelchair traffic data from car traffic data may add
many biases.
Even if this method is innovative, it highlights the need for dedicated data collection on other locomotion modes than
cars.
The second implication is scalability, as the performance on the METR-LA dataset suggests potential scalability.
But the differences in results between cars, wheelchairs and both indicate otherwise.
Lastly, the model generalization is a limitation.
While the model performs well on the METR-LA dataset, its generalization to other areas remains untested.
Further efforts are needed to evaluate the model across varied datasets.
\vspace{1em}

We propose several directions for future work.
\begin{itemize}
    \item \textbf{Data Collection}:
    Conducting new data collection efforts could be beneficial for the research community.
    Data from other countries or cities could allow more tests for better models.
    \item \textbf{Model Improvement}:
    Further improvements to the model architecture could enhance its performance on multi-modal traffic prediction.
    Adding other techniques or dividing the prediction task between multiple models are possible directions of research.
    \item \textbf{Adding Accessibility Features}:
    Adding more accessibility features to the data, such as public transport accessibility, could emphasize the social
    contribution of our study.
    \item \textbf{Model Comparison}:
    We could improve our understanding of the model's performance by comparing it to other models, and measuring its
    performance on different metrics and horizons.
\end{itemize}
\vspace{1em}
