The need for itinerary recommendation systems has been significantly met by large platforms like Google Maps, Waze, and
Apple Maps.
They offer a wide range of functionalities, including itineraries for mobility-impaired users.
However, if a user needs to get around using a wheelchair, the options are much more limited.
For instance, any of the broadly known itinerary recommendation systems are capable of computing the time to go
somewhere by using solely the wheelchair as transportation mode.
Additionally, these companies collect private data, raising concerns about user privacy.
To address these issues, there is a growing need for an open-source model that can provide comprehensive itinerary
recommendations for all modes of transportation, including car, bike, wheelchair and walking.
\vspace{1em}

A graph is defined as follows: $G = (V, E)$, where $V$ is the set of nodes and $E$ is the set of edges.
Graphs allows complex representations of data outside Euclidean space.
Common Convolutional Neural Networks (CNNs) assume that the data is in a grid-like structure, this works well for images
but not for our case.
As traffic occurs on a road network that can be represented as a graph, we need a Graph Neural Network (GNN).
These type of neural networks allow us to perform complex operations on graph structures.
These operations enable the capture of spatial and temporal dependencies in the data, which is crucial for a problem
like traffic prediction.
\vspace{1em}

Multi-modal traffic prediction faces multiple challenges.
The first one is about data sources, as data collection on multiple modes of transportation within the same area has
never been done before.
The artificial intelligence model is the second challenge.
Managing various locomotion modes and capturing the temporal and spatial dependencies of traffic is a challenging task.
Finally, the third challenge is about scalability.
Scalability is needed for accurate real-time traffic prediction.
\vspace{1em}

To demonstrate the feasibility of multi-modal traffic prediction, we propose a model that can take in consideration cars
and wheelchairs as transportation modes.
Wheelchair users often faces unique challenges in mobility, such as inaccessible roads or sidewalks.
Given such challenges there is a need for a model that is capable of providing predictions for multiple kinds of
transportation including the wheelchairs.
\vspace{1em}

This paper proposes an adaptation of the Diffusion Convolutional Recurrent Neural Network (DCRNN) model~\cite{DCRNN}
to handle the complexity of multi-modal traffic prediction.
The DCRNN model take temporal sequences of traffic data from the METR-LA dataset~(\ref{subsec:derivation}
) and predicts the next sequence.
Our approach includes:
\begin{itemize}
    \item \textbf{Model Adaptation}:
    Enhancing the DCRNN model to take into account specifically cars and wheelchairs.
    \item \textbf{Data Derivation}:
    Developing a method to derive wheelchair traffic data from car traffic data in the METR-LA dataset.
\end{itemize}
\vspace{1em}
