Graph Neural Networks (GNNs) are often used traffic prediction tasks, due to their ability to model complex
spatial-temporal data.
The groundwork~\cite{DCRNN}
introduces a combination of diffusion convolution and gated recurrent units to capture both spatial and temporal
dependencies of traffic data.
Other studies show the potential of GNNs in complex tasks,~\cite{Wu2019A}
introduces categories of GNNs and surveys their applications.
Multi-modal traffic prediction is achieved by combining data from multiple sources like cars or bikes.
\cite{zhang2017deep} uses a deep spatio-temporal residual network to predict crowd flows.
They use data from taxis and bikes, proving the usefulness of multi-modal data to enhance performances.
Accessibility for mobility-impaired people is a growing area of research.
\cite{repetto2022developing} developed an accessibility index to evaluate public transports.
They emphasize the importance of integrating accessibility to our systems.
\vspace{1em}
