\subsection{Introduction}\label{subsec:introduction}
\textcolor{gray}
{Sets the stage for the mission, providing context and significance without going into specific project details.}

The primary purpose of my internship was to undertake a research project and produce a research paper on the assigned
topic.
The internship took place in a research laboratory associated with my engineering school, where I, along with
nine other students, was given a unique research topic.
Although the internship was positioned within an academic setting, the primary objective appeared to be more focused on
allowing us to independently explore and develop our research skills rather than contributing directly to the
laboratory's broader research goals.
This experience served as an opportunity for us to gain a deeper understanding of the research process, develop critical
thinking, and enhance our ability to work autonomously.

\subsection{Company Description}\label{subsec:company-description}
\textcolor{gray}
{Focuses on the company as a whole, giving background information and context that is broader than your specific
internship. Narrows down to the specific department,
    explaining its role within the company and setting up the context for your specific tasks.}


ECE Paris is a well-established engineering school, founded in 1919, with a long-standing tradition of excellence in
science and technology education.
The school is committed to producing well-rounded engineers equipped with the skills necessary for innovation,
international collaboration, and entrepreneurship.

Within this academic environment, Lyrids serves as the dedicated research laboratory of ECE Paris.
Established to promote high-level research, Lyrids focuses on several key areas including artificial intelligence, data
science,
cybersecurity, and connected systems.
The laboratory is organized into specialized research groups, each contributing to the advancement of knowledge in their
respective fields.
Lyrids plays a crucial role in integrating research into the academic curriculum, allowing students to engage with
cutting-edge projects and develop their research skills in a practical, hands-on environment.

\subsection{Purpose of the Internship}\label{subsec:purpose-of-the-internship}
\textcolor{gray}
{Details the objectives and goals specific to your internship experience, linking it to the company's broader mission.}

The primary objective of my internship was to propose a viable solution to the problem posed by my research topic.
This experience was designed not only to challenge me intellectually but also to foster my ability to work autonomously.
Throughout the internship, I gained significant hands-on experience, particularly in coding with \LaTeX{}, conducting
in-depth literature reviews, and strengthening my skills in neural network training and Python programming.

My expectation was to delve deeper into the field of artificial intelligence, and this internship fully met that goal.
The opportunity to focus on AI allowed me to explore and develop expertise in a domain that is not only fascinating but
also pivotal for my future career aspirations.
By engaging in research, I was able to acquire and refine a set of skills that will be highly beneficial as I continue
to pursue a career in engineering and technology.

\subsection{Context of the Internship}\label{subsec:context-of-the-internship}
\textcolor{gray}
{Describes the environment and any notable circumstances that influenced your internship, distinct from the daily tasks
and projects.}

The internship took place in a collaborative environment, with all ten interns working together in a classroom at the
school, conveniently located near the laboratory office.
This setup allowed for easy access to our mentors whenever guidance was needed.
While we didn't have access to many specialized resources, we made extensive use of Google Scholar and a specific
website that provides research papers, to which the school had a subscription.

The internship began with a loosely defined research topic, which required us to refine and narrow our focus as we
progressed.
Regular meetings with my mentor and another faculty member were crucial in planning and directing my research,
especially as the initial topic evolved significantly over time.
My mentor was supportive, particularly when I encountered challenges, and the collaborative atmosphere among the interns
in the shared space was a great advantage.
We frequently assisted each other, which enhanced the overall learning experience.

Working fully on-site was particularly beneficial for me, as I find it difficult to maintain focus when working from
home.
The structured environment of the school helped me stay productive and engaged with my work.

\subsection{Role and Responsibilities}\label{subsec:role-and-responsibilities}
\textcolor{gray}
{Specifically outlines what you did on a day-to-day basis, your tasks, and your contributions.}
During my internship, my responsibilities were structured into several key stages, each aimed at developing my research
project comprehensively.
The first stage involved conducting a thorough literature review, where I read various research papers to gain a deeper
understanding of the topic and familiarize myself with the relevant coding environment,particularly Python and its
associated packages.

Following the literature review, I moved on to the development stage, where my primary task was to propose and develop a
new solution to the problem presented by my research topic.
This phase required a combination of critical thinking,coding, and experimentation to arrive at a feasible solution.

The final major task was to document my work by writing a research paper detailing the problem, the methodology I
employed, the solution I developed, and the results of my work.
This paper was a significant deliverable, and after submitting it before the end of the internship, I received feedback
for revisions, which I am currently implementing.

In addition to these stages, I participated in a midterm presentation where I, along with the other interns, presented
our progress to the mentors.
This presentation served as a checkpoint, allowing us to receive constructive feedback and refine our approach as
needed.

Throughout the internship, all tasks and responsibilities were carried out individually, though we benefited from a
collaborative environment where we could assist each other.
While there were no specific performance goals to meet, the focus was on contributing to research, where even negative
or unexpected results were valuable.

